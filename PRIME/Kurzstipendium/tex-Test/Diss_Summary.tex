%%%%%%%%%%%%%%%%%%%%%%%%%%%%%%%%%%%%%%%%%
% "ModernCV" CV and Cover Letter
% LaTeX Template
% Version 1.3 (29/10/16)
%
% This template has been downloaded from:
% http://www.LaTeXTemplates.com
%
% Original author:
% Xavier Danaux (xdanaux@gmail.com) with modifications by:
% Vel (vel@latextemplates.com)
%
% License:
% CC BY-NC-SA 3.0 (http://creativecommons.org/licenses/by-nc-sa/3.0/)
%
% Important note:
% This template requires the moderncv.cls and .sty files to be in the same 
% directory as this .tex file. These files provide the resume style and themes 
% used for structuring the document.
%
%%%%%%%%%%%%%%%%%%%%%%%%%%%%%%%%%%%%%%%%%

%----------------------------------------------------------------------------------------
%	PACKAGES AND OTHER DOCUMENT CONFIGURATIONS
%----------------------------------------------------------------------------------------

\documentclass[11pt,a4paper,sans]{moderncv} % Font sizes: 10, 11, or 12; paper sizes: a4paper, letterpaper, a5paper, legalpaper, executivepaper or landscape; font families: sans or roman

\moderncvstyle{casual} % CV theme - options include: 'casual' (default), 'classic', 'oldstyle' and 'banking'
\moderncvcolor{blue} % CV color - options include: 'blue' (default), 'orange', 'green', 'red', 'purple', 'grey' and 'black'

\usepackage{lipsum} % Used for inserting dummy 'Lorem ipsum' text into the template

\usepackage[scale=0.75]{geometry} % Reduce document margins
\setlength{\hintscolumnwidth}{0.205\textwidth} % Uncomment to change the width of the dates column
%\setlength{\makecvtitlenamewidth}{10cm} % For the 'classic' style, uncomment to adjust the width of the space allocated to your name

%----------------------------------------------------------------------------------------
%	NAME AND CONTACT INFORMATION SECTION
%----------------------------------------------------------------------------------------

\firstname{Christine} % Your first name
\familyname{Nussbaum} % Your last name

% All information in this block is optional, comment out any lines you don't need
\title{Dissertation Zusammenfassung} 
%\address{Raum 107, Am Steiger 3/1}{07743 Jena}
%\mobile{+49 (0) 176 45629376}
%\phone{+49 (0) 3641 9 45934}
%\email{christine.nussbaum@uni-jena.de}
%\homepage{www.allgpsy.uni-jena.de/christine-nussbaum/}{www.allgpsy.uni-jena.de/christine-nussbaum} % The first argument is the url for the clickable link, the second argument is the url displayed in the template - this allows special characters to be displayed such as the tilde in this example
%\extrainfo{additional information}
%\photo[100pt][0.4pt]{c_nussbaum_sw.jpg} % The first bracket is the picture height, the second is the thickness of the frame around the picture (0pt for no frame)

%----------------------------------------------------------------------------------------

\begin{document}
\makecvtitle

\section{Hintergrund und Fragestellung}
Die menschliche Stimme ist ein wichtiger Transmitter sozialer Signale, auch über das gesprochene Wort hinaus. Besonders unsere Emotionen werden durch den Klang unserer Stimme hörbar. Eine adäquate und effiziente Wahrnehmung von Emotionen in der menschlichen Stimme ist dabei von großer Bedeutung im alltäglichen Miteinander. Zu verstehen, wie wir Menschen Emotionen in der Stimme wahrnehmen und welche individuellen Unterschiede es hinsichtlich dieser Fähigkeit gibt, ist daher von großem grundlagenwissenschaftlichem Interesse. \\
Vorangegangene Untersuchungen haben gezeigt, dass die meisten Menschen in der Lage sind, Emotionen in der Stimme korrekt zu erkennen. Dieser Erkennungsleistung liegt eine sehr effiziente und automatische Verarbeitung der akustischen Stimmparameter zugrunde, die sich in Abhängigkeit von emotionalen Zuständen verändern. Bei Ärger, beispielsweise, bekommt unsere Stimme meist eine höhere Tonlage, eine deutlich schärfere Klangfarbe und klingt lauter. In meiner Arbeit habe ich moderne Voice-Morphing-Technologie eingesetzt, um den emotionalen Ausdruck einzelner Stimmenparameter präzise zu kontrollieren und deren kausale Einflüsse zu quantifizieren. Die erste Fragestellung meiner Arbeit lautete: (1) Welchen Beitrag leisten verschiedene akustische Signale (speziell die Tonhöhe und die Klangfarbe) zur Wahrnehmung von Emotionen in der Stimme? \\
In einem zweiten Schritt habe ich individuelle Unterschiede untersucht, speziell im Hinblick auf auditorische Expertise und Musikalität. Aus der Literatur geht hervor, dass MusikerInnen im Vergleich zu Nicht-MusikerInnen etwas besser darin sind, Emotionen in der Stimme zu erkennen. Warum das so ist, und welche Mechanismen dem zugrunde liegen, ist jedoch nicht bekannt. Mich hat besonders interessiert, welche Rolle die auditorische Sensitivität bei MusikerInnen spielt und inwieweit MusikerInnen die akustischen Parameter in der Stimme anders oder effizienter nutzen, um Emotionen zu erkennen. Meine zweite Fragestellung lautete daher: (2) Wie wirkt sich die Musikalität auf die Verarbeitung von emotionalen Stimmenparametern aus? \\
Zu guter Letzt stellt die vorliegende Arbeit eine kritische Auseinandersetzung mit der Voice-Morphing Technologie dar, welche ich im Rahmen dieser Doktorarbeit erstmals für die akustische Manipulation von Emotionen und in Kombination mit einer Elektroenzephalogramm(EEG)-Messung eingesetzt habe. Obwohl diese Technologie ein enormes Potenzial für die systematische Erforschung emotionaler Klangparameter hat, stellte sich heraus, dass die manipulierten Stimmen mitunter verzerrt klingen, was deren Validität einschränken kann. Mit dieser Problematik habe ich mich in meiner dritten Fragestellung auseinandergesetzt: (2) Ist das (parameter-spezifische) Voice-Morphing ein geeignetes Instrument zur Untersuchung der Emotionsverarbeitung in menschlichen Stimmen?


\section{Methodisches Vorgehen}

Das methodische Rückgrat meiner Dissertation bildet das parameter-spezifische Voice Morphing. Diese Technik habe ich genutzt, um kurze stimmliche Äußerungen zu erstellen, die vier Emotionen (Freude, Genuss, Angst und Trauer) nur durch die Tonhöhe, nur durch die Klangfarbe oder durch beides ausdrücken. Diese wurden anschließend nicht nur für Verhaltensexperimente, sondern auch für zwei Untersuchungen mit Elektroenzephalogramm (EEG) eingesetzt. 
Meine drei Fragestellungen werden in insgesamt 5 Teilen adressiert, die inzwischen alle als Publikationen veröffentlicht wurden: \\

1.	ein systematisches Review über den Zusammenhang zwischen Musikalität und Emotionswahrnehmung in der menschlichen Stimme 
2.	eine empirische Online-Ratingstudie zur Validierung des Stimulus-Materials, welches mithilfe von Voice-Morphing erzeugt wurde. 
3.	eine empirische EEG-Studie, die sowohl verhaltensbezogene und elektrophysiologische Daten umfasst und in der die Rolle von Tonhöhe und Klangfarbe für die Wahrnehmung von Emotionen in der Stimme untersucht wurde. 
4.	eine empirische Online-Studie, in der die Emotionserkennungsleistungen von je knapp 40 MusikerInnen und Nicht-MusikerInnen verglichen wurde. Von besonderem Interesse war, wie beide Gruppen die Tonhöhe und Klangfarbe nutzen, um Emotionen wahrzunehmen.
5.	einer empirischen EEG-Studie, in der untersucht wurde, inwieweit sich die Unterschiede zwischen MusikerInnen und Nicht-MusikerInnen auf neuronaler Ebene widerspiegelt. 

\section{Wichtigste Ergebnisse}

\paragraph{(1)	Welchen Beitrag leisten verschiedene akustische Signale (speziell die Tonhöhe und die Klangfarbe) zur Wahrnehmung von Emotionen in der Stimme?}

Es hat sich gezeigt, dass sowohl die Tonhöhe als auch die Klangfarbe emotionale Informationen transportieren, welche von Hörenden flexibel genutzt werden können um auf die ausgedrückte Emotion zu schließen.  Ihre spezifische Bedeutsamkeit hängt jedoch von der Emotion ab: bei Emotionen mit großer Intensität, wie Freude oder Angst, spielt die Tonhöhe eine deutlich wichtigere Rolle als die Klangfarbe. Bei etwas weniger intensiven Emotionen, wie Trauer oder Genuss, ist der Einfluss von Tonhöhe und Klangfarbe ausgeglichener. Ein solches Muster spiegelt sich auch auf Ebene der neuronalen Verarbeitung wider. 

\paragraph{(2)	Wie wirkt sich die Musikalität auf die Verarbeitung von emotionalen Stimmenparametern aus?}

Zunächst konnte ich replizieren, dass MusikerInnen Emotionen in der Stimme etwas besser erkennen können als Nicht-MusikerInnen. Vor allem jedoch weisen die Daten auf eine besondere Bedeutung der Tonhöhe für MusikerInnen hin: MusikerInnen zeigten besser Erkennungsleistungen als Nicht-MusikerInnen, wenn die Emotionen nur durch die Tonhöhe ausgedrückt wurden, aber nicht, wenn sie durch die Klangfarbe ausgedrückt wurden. Des Weiteren zeigte sich, dass der Zusammenhang zwischen musikalischem Hörvermögen und der Emotionserkennung in Stimmen unabhängig von formaler Musikausbildung bestehen bleibt, was auf eine Prädisposition zur effizienten Nutzung akustischer Signale bei musikalisch begabten Personen hindeutet.

\paragraph{(3)	Ist das (parameter-spezifische) Voice-Morphing ein geeignetes Instrument zur Untersuchung der Emotionsverarbeitung in menschlichen Stimmen?}

Meine Daten haben gezeigt, dass Voice Morphing die akustische Qualität der Stimmen zwar beeinträchtigen kann, sich die Wahrnehmung der Emotionen jedoch als bemerkenswert robust gegenüber diesen Verzerrungen erweist. Insgesamt präsentiert diese Arbeit damit überzeugende Befunde, dass Voice-Morphing ein valides Instrument für die Erforschung von Emotionen in der Stimme darstellt, wenn es mit einem kritischen Bewusstsein für seine Grenzen und Probleme zum Einsatz gebracht wird.

\section{Relevanz im Fachgebiet}

Die vorliegende Dissertation liefert wichtige neue Erkenntnisse über die Wahrnehmung
von Emotionen in der menschlichen Stimme, sowohl auf empirischer als auch auf konzeptioneller
Ebene. Die zentralen Beiträge beziehen sich dabei auf die Rolle der zugrundeliegenden akustischen
Parameter, die elektrophysiologischen Korrelate, und den Einfluss individueller Unterschiede mit
einem spezifischen Fokus auf Musikalität. Auf diese Weise trägt diese Arbeit zum Verständnis
mehrerer komplexer und wichtiger Eigenschaften bei, welche uns als Menschen ausmachen:
dem Gebrauch unserer Stimmen, dem Ausdruck unserer Emotionen und unserer Fähigkeit, zu
Musizieren. 



\end{document}